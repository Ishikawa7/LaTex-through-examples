\documentclass{article}

\usepackage[margin=1in]{geometry}

\title{Tables Floats and the Tabular Enviroment}
\author{Davide Grimaldi}
\date{}

\begin{document}
    \maketitle
    \section{Introduction}
        Tables examples

        \subsection{The Tabular Enviroment}
            This is the tabular Enviroment for the table~\ref{tab:grades}
            \vspace*{.5cm} %vertival space
            \\
            % we can use \smallskip or \bigskip commands for vertical spacing
            \begin{tabular}{|c|c|} %two "l" one left line for each column, "c" for centered columns, "|" for vertical bars
                \hline %orizontal bar
                Name & Your Name Here \\
                \hline
                Subject & \LaTeX \\
                \hline
                Grade & 30/30 \\
                \hline
            \end{tabular}
        
        % in LaTex a table is a float element: LaTex try to put that element in your document where it looks nicer    
        \subsection{Floats: Table}
            %begin create a new environment
            \begin{table}[htbp] % [preference for positioning] h=here t=top b=bottom p=page(next page) 
                \caption{My Gradebook}
                \begin{center}
                    %precedent tabular 
                    \begin{tabular}{|l|l|} 
                        \hline
                        Name & Your Name Here \\
                        \hline
                        Subject & \LaTeX \\
                        \hline
                        Grade & 30/30 \\
                        \hline
                    \end{tabular}
                \end{center}
                \label{tab:grades} %label used for referencing this table (see down)
            \end{table}

            \subsection{References}
                See my grades table~\ref{tab:grades}.

    \section{Conclusion}
        This is the basics for creating tables
\end{document}